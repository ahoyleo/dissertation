\abstract{Data mining, also referred to as knowledge discovery in databases (KDD), is the nontrivial extraction of implicit, previously unknown, and potentially useful information from data. The measure of what is meant by ``useful" to the user is dependent on the user as well as the domain within which the data mining system is being used. Therefore, the role of domain knowledge in the discovery process is essential. However, the synergy between domain knowledge and data mining is still at a rudimentary level. This motivates us to develop a framework for explicit incorporation of domain knowledge in a data mining system so that insights can be drawn from both data and domain knowledge. We call such technology ``semantic data mining." %We showcase the technology on a particular mining problem that aims to find indirect associations in this dissertation.

Recent research in knowledge representation, particularly in the area of W3C's Semantic Web that seeks to embed semantic content in web pages, has led to mature standards such as the Web Ontology Language (OWL) for authoring ontologies. An ontology is an explicit specification of a conceptualization. Today, Semantic Web ontologies have become a key technology for intelligent knowledge processing, providing a framework for sharing conceptual models about a domain. We make use of ontologies as a means to encode domain knowledge in this dissertation.

The OWL ontology language is built on the W3C's Resource Description Framework (RDF) that provides a simple model to describe information resources as a graph. At the same time, there has been a surge of interest in tackling the problem of mining semantically rich datasets, where objects are linked in a number of ways. In fact, many datasets of interest today are best described as a linked collection, or a graph, of interrelated objects. We notice that the interface between domain knowledge and data mining can be possibly achieved by using graph representations in which distinct sorts of knowledge %that has been traditionally differently represented
can be structured in a unified manner. Therefore, we explore a graph-based approach for modeling both knowledge and data, and for analyzing the combined information source from which insight can be drawn systematically.

In summary, this dissertation presents three contributions in semantic data mining. First, we develop a solution based on metaheuristic optimization to automatically resolve schema heterogeneities as well as to achieve pattern comparison for meta-analysis when mining tasks require accessing disparate, heterogeneous data sources. Second, we describe how a graph interface for both knowledge representation and data mining can be structured. This is achieved by employing the RDF model given the fact that RDF allows a combined specification of both schema and data structured under this schema. We propose to use the RDF hypergraph/bipartite graph model as the unified representation for both data and domain knowledge. And finally, we describe several graph theoretic analysis approaches for mining the combined information source. We showcase the utility of the methods on finding semantically associated itemsets, a particular case of the frequent pattern mining. We hope these novel contributions can lead to the development of new principles towards semantic data mining.} 