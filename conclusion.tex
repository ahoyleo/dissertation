This dissertation proposes the framework of semantic data mining, a novel direction for the field of data mining that focuses on systematic incorporation of domain knowledge. The enabling technology is based on two contributions presented in the dissertation. First, we develop a graph-based formalism that allows a coherent representation for both the data and the domain knowledge. The key concepts of the approach are bridging the data and the ontology by semantic annotation and employing the RDF bipartite graph as the unified representation. Second, we demonstrate analysis techniques that can be carried out based on the RDF bipartite graph to tackle common data mining tasks, such as the frequent itemset mining, while at the same time leveraging domain knowledge to enhance the performance. For this purpose, several graph-based similarity measures are provided as the key components in the mining algorithms and their trade-offs are studied. The concept of random walk inside the graph while traversing and calculating similarities among nodes are used in designing these measures. 

This dissertation also presents the details of some case studies that have validated the hypotheses used in designing the graph-based semantic data mining framework.

\section{Future Work}
Semantic data mining is a new field, many interesting research directions related to it are yet to be explored. The research work presented in this dissertation can be extended in several directions. The following are the main directions we have identified.

\subsection{Learning Weights Automatically}
\subsection{Handling Numeric Values}
\subsection{Scalability Issues}
\subsection{New Ways of Representing Complex Domain Semantics}

\section{Concluding Remarks}